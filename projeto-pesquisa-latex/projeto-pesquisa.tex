%%%%%%%%%%%%%%%%%%
%%%   Template para utilização   %%%
%%%%%%%%%%%%%%%%%%

%Para melhor utilização da ferramenta, utilize o pdfLaTeX+MakeIndex+BibTeX%

\documentclass[12pt,openright,oneside,a4paper,english,french,spanish,brazil]{unifil}

\titulo{Título} % Título da monografia
\autor{Seu Nome} % Nome do autor
\instituicao{Centro Universitário Filadélfia}
\local{Londrina}
\data{2019} % Ano de publicação
\preambulo{Nome do Curso} % Nome do curso
\orientador{Nome do Orientador} % Nome do orientador

\begin{document}

\frenchspacing

%%%%%%%%%%%%%%%%%%%%%%%%%%%%
%% Elementos pré-textuais %%
%%%%%%%%%%%%%%%%%%%%%%%%%%%%

\pretextual

\imprimircapa
\makeatletter
\renewcommand{\folhaderostocontent}{
\begin{center}
{\ABNTEXchapterfont\bfseries\MakeTextUppercase{\imprimirautor}}
\vspace*{5cm}
\begin{center}
\ABNTEXchapterfont\bfseries\normalsize\MakeTextUppercase{\imprimirtitulo}
\end{center}
\vspace*{1cm}
\abntex@ifnotempty{\imprimirpreambulo}{%
\hspace{.45\textwidth}
\begin{minipage}{.5\textwidth}
\SingleSpacing
 Trabalho de Dissertação apresentado ao \imprimirinstituicao como parte dos requisitos para obtenção de graduação em \imprimirpreambulo.
{\textnormal{\imprimirorientadorRotulo~\imprimirorientador.}}
\end{minipage}%
}%
\vspace*{\fill}
{\bfseries\large\imprimirlocal}
\par
{\bfseries\large\imprimirdata}
\vspace*{1cm}
\end{center}
}
\makeatother

\imprimirfolhaderosto

\clearpage{\pagestyle{empty}\cleardoublepage}
%%Altere as informações do resumo%%
\noindent{SOBRENOME; NOME, D. \textbf{\imprimirtitulo}. Trabalho de Conclusão de Curso (Graduação) - \imprimirinstituicao. \imprimirlocal, \imprimirdata.}
\par
\begin{resumo}
	%%Escreva seu resumo na língua vernácula aqui%%
\lipsum[5]
\vspace{\onelineskip} 
	%%Adicione as palavras chaves após os dois pontos '':''%%
\noindent
\textbf{Palavras-chaves}: 3 ou mais.

\end{resumo}


%\par
%\vspace{11cm}

\tableofcontents*

  \setlength\absleftindent{0cm}
  \setlength\absrightindent{0cm}
  
  %fonte do ambiente%
  \abstracttextfont{\normalfont\normalsize}

  %intentação e espaçamento entre parágrafos%
  \setlength{\absparindent}{0pt}
  \setlength{\absparsep}{18pt}

%\pdfbookmark[0]{\listfigurename}{lof}
%\listoffigures*
%\cleardoublepage

\textual

\renewcommand{\ABNTEXchapterfont}{\fontfamily{cmr}\fontseries{b}\selectfont}
\renewcommand{\ABNTEXchapterfontsize}{\Large}

\renewcommand{\ABNTEXsectionfont}{\uppercase{\fontfamily{cmr}\fontseries{b}\selectfont}}
\renewcommand{\ABNTEXsectionfontsize}{\large}

\chapter{Introdução}%%Inserir título do capítulo (nível 1)

%%Conteúdo
\lipsum[2]

\section{Seção}%%Inserir seção (nível 2)

\lipsum[2]

\lipsum[2]

%%Conteúdo

\subsection{Subseção}%%Inserir subseção (nível 3)

\lipsum[2]

\lipsum[2]

%%Conteúdo

\subsubsection{Subsubseção}%%Inserir subsubseção (nível 4)

\lipsum[3]

%%Conteúdo

\subsubsubsection{Subsubsubseção}%%Inserir subsubsubseção (nível 5)

\lipsum[2]

%%Conteúdo
\chapter{Problemática da pesquisa e metodologia}

Descrição do problema de pesquisa a ser abordado, hipótese e pré-evidências (tanto do problema quanto para a hipótese).

Metodologia a ser utilizada para verificar a hipótese, com justificativa.

\chapter{Resultados esperados}

Dissertar sobre os desdobramentos dos possíveis resultados do teste de sua hipótese. Pré-análise.

\section{Limitações do trabalho}

Dada as questões de pesquisa e a metodologia, as vezes é necessário clarificar que algumas dessas questões não podem ou não serão respondidas. Em outros casos, há questões muito próximas às abordadas, é útil clarificar e explicar porque ficaram de fora.

\chapter{Breve estado da arte}

Descrição do estado da arte do tema. O estudante deve demonstrar conhecimento e embasamento. Entre 5 a 10 referências.

\chapter{Cronograma}

Faça um cronograma de atividades por semana de trabalho.
\begin{itemize}
\item Somente atividades da metodologia e etapas da escrita da monografia.
\item Não deve repetir as atividades da disciplina de TCC.
\item É necessário ser definido junto ao orientador.
\end{itemize}

\chapter{Leitura e fichamento da bibliografia}

Neste capítulo, registrem todo o trabalho de fichamento de bibliografia, utilizando a seguinte estrutura.

\section{Título da obra bibliográfica 1}

\begin{description}
\item[Autores:] Fulano \emph{et al.}
\item[Ano:] 2017
\item[Relevância:] Qualis, índice $H^*$ ou quantidade de citações.
\end{description}

Faça um dicionário, anote todos os termos desconhecidos seguidos de seus significados recém aprendidos.

Faça uma lista de palavras-chave descobertas neste artigo e que foram relevantes para novas buscas de levantamento bibliográfico.

Escreva uma resenha em prosa daquilo que você apreendeu da obra em questão.

\section{Título da obra bibliográfica 2}

\begin{description}
\item[Autores:] Ciclano \emph{et al.}
\item[Ano:] 2015
\item[Relevância:] Qualis, índice $H^*$ ou quantidade de citações.
\end{description}

Faça um dicionário, anote todos os termos desconhecidos seguidos de seus significados recém aprendidos.

Faça uma lista de palavras-chave descobertas neste artigo e que foram relevantes para novas buscas de levantamento bibliográfico.

Escreva uma resenha em prosa daquilo que você apreendeu da obra em questão.

\cleardoublepage

\postextual

%%Colocar as referências conforme as normas da ABNT, somente as utilizadas no trabalho e presentes neste manuscrito.

\bibliography{bibliografia}{}

% \begin{thebibliography}{99}

% \bibitem{ABNTEX2:2014}
% {ABNTEX2; ARAUJO, L. C. \textbf{A classe abntex2}: Documentos técnicos e científicos brasileiros compatíveis com as normas ABNT. Sine loco, v. 1.9.2; 2014.}.

% \bibitem{Biazin:2008}
% {BIAZIN, D. T. \textbf{Normas da ABNT e padronização de trabalhos acadêmicos}. Londrina: Instituto Filadélfia de Londrina; 2008.}

% \bibitem{Buneman:2011}
% {BUNEMAN, P.; CHENEY, J.; LINDLEY, S. et al. \textbf{DBWiki}: A Structured Wiki for Curated Data and Collaborative Data Management. Athens: SIGMOD’11; 2011.}

% \bibitem{Wikibooks:2014}
% {WIKIBOOKS. \textbf{LaTeX}: The Free Textbook Project. Disponível em: <http://en.wikibooks.org/wiki/LaTeX>. Acesso em: 09 abr. 2014.}

% \end{thebibliography}

\end{document}